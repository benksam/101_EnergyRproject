\documentclass[]{article}
\usepackage{lmodern}
\usepackage{amssymb,amsmath}
\usepackage{ifxetex,ifluatex}
\usepackage{fixltx2e} % provides \textsubscript
\ifnum 0\ifxetex 1\fi\ifluatex 1\fi=0 % if pdftex
  \usepackage[T1]{fontenc}
  \usepackage[utf8]{inputenc}
\else % if luatex or xelatex
  \ifxetex
    \usepackage{mathspec}
  \else
    \usepackage{fontspec}
  \fi
  \defaultfontfeatures{Ligatures=TeX,Scale=MatchLowercase}
\fi
% use upquote if available, for straight quotes in verbatim environments
\IfFileExists{upquote.sty}{\usepackage{upquote}}{}
% use microtype if available
\IfFileExists{microtype.sty}{%
\usepackage{microtype}
\UseMicrotypeSet[protrusion]{basicmath} % disable protrusion for tt fonts
}{}
\usepackage[margin=1in]{geometry}
\usepackage{hyperref}
\hypersetup{unicode=true,
            pdftitle={Income, Energy Consumption and CO2-Emissions with R Notebook},
            pdfauthor={SAB},
            pdfborder={0 0 0},
            breaklinks=true}
\urlstyle{same}  % don't use monospace font for urls
\usepackage{color}
\usepackage{fancyvrb}
\newcommand{\VerbBar}{|}
\newcommand{\VERB}{\Verb[commandchars=\\\{\}]}
\DefineVerbatimEnvironment{Highlighting}{Verbatim}{commandchars=\\\{\}}
% Add ',fontsize=\small' for more characters per line
\usepackage{framed}
\definecolor{shadecolor}{RGB}{248,248,248}
\newenvironment{Shaded}{\begin{snugshade}}{\end{snugshade}}
\newcommand{\KeywordTok}[1]{\textcolor[rgb]{0.13,0.29,0.53}{\textbf{#1}}}
\newcommand{\DataTypeTok}[1]{\textcolor[rgb]{0.13,0.29,0.53}{#1}}
\newcommand{\DecValTok}[1]{\textcolor[rgb]{0.00,0.00,0.81}{#1}}
\newcommand{\BaseNTok}[1]{\textcolor[rgb]{0.00,0.00,0.81}{#1}}
\newcommand{\FloatTok}[1]{\textcolor[rgb]{0.00,0.00,0.81}{#1}}
\newcommand{\ConstantTok}[1]{\textcolor[rgb]{0.00,0.00,0.00}{#1}}
\newcommand{\CharTok}[1]{\textcolor[rgb]{0.31,0.60,0.02}{#1}}
\newcommand{\SpecialCharTok}[1]{\textcolor[rgb]{0.00,0.00,0.00}{#1}}
\newcommand{\StringTok}[1]{\textcolor[rgb]{0.31,0.60,0.02}{#1}}
\newcommand{\VerbatimStringTok}[1]{\textcolor[rgb]{0.31,0.60,0.02}{#1}}
\newcommand{\SpecialStringTok}[1]{\textcolor[rgb]{0.31,0.60,0.02}{#1}}
\newcommand{\ImportTok}[1]{#1}
\newcommand{\CommentTok}[1]{\textcolor[rgb]{0.56,0.35,0.01}{\textit{#1}}}
\newcommand{\DocumentationTok}[1]{\textcolor[rgb]{0.56,0.35,0.01}{\textbf{\textit{#1}}}}
\newcommand{\AnnotationTok}[1]{\textcolor[rgb]{0.56,0.35,0.01}{\textbf{\textit{#1}}}}
\newcommand{\CommentVarTok}[1]{\textcolor[rgb]{0.56,0.35,0.01}{\textbf{\textit{#1}}}}
\newcommand{\OtherTok}[1]{\textcolor[rgb]{0.56,0.35,0.01}{#1}}
\newcommand{\FunctionTok}[1]{\textcolor[rgb]{0.00,0.00,0.00}{#1}}
\newcommand{\VariableTok}[1]{\textcolor[rgb]{0.00,0.00,0.00}{#1}}
\newcommand{\ControlFlowTok}[1]{\textcolor[rgb]{0.13,0.29,0.53}{\textbf{#1}}}
\newcommand{\OperatorTok}[1]{\textcolor[rgb]{0.81,0.36,0.00}{\textbf{#1}}}
\newcommand{\BuiltInTok}[1]{#1}
\newcommand{\ExtensionTok}[1]{#1}
\newcommand{\PreprocessorTok}[1]{\textcolor[rgb]{0.56,0.35,0.01}{\textit{#1}}}
\newcommand{\AttributeTok}[1]{\textcolor[rgb]{0.77,0.63,0.00}{#1}}
\newcommand{\RegionMarkerTok}[1]{#1}
\newcommand{\InformationTok}[1]{\textcolor[rgb]{0.56,0.35,0.01}{\textbf{\textit{#1}}}}
\newcommand{\WarningTok}[1]{\textcolor[rgb]{0.56,0.35,0.01}{\textbf{\textit{#1}}}}
\newcommand{\AlertTok}[1]{\textcolor[rgb]{0.94,0.16,0.16}{#1}}
\newcommand{\ErrorTok}[1]{\textcolor[rgb]{0.64,0.00,0.00}{\textbf{#1}}}
\newcommand{\NormalTok}[1]{#1}
\usepackage{graphicx,grffile}
\makeatletter
\def\maxwidth{\ifdim\Gin@nat@width>\linewidth\linewidth\else\Gin@nat@width\fi}
\def\maxheight{\ifdim\Gin@nat@height>\textheight\textheight\else\Gin@nat@height\fi}
\makeatother
% Scale images if necessary, so that they will not overflow the page
% margins by default, and it is still possible to overwrite the defaults
% using explicit options in \includegraphics[width, height, ...]{}
\setkeys{Gin}{width=\maxwidth,height=\maxheight,keepaspectratio}
\IfFileExists{parskip.sty}{%
\usepackage{parskip}
}{% else
\setlength{\parindent}{0pt}
\setlength{\parskip}{6pt plus 2pt minus 1pt}
}
\setlength{\emergencystretch}{3em}  % prevent overfull lines
\providecommand{\tightlist}{%
  \setlength{\itemsep}{0pt}\setlength{\parskip}{0pt}}
\setcounter{secnumdepth}{0}
% Redefines (sub)paragraphs to behave more like sections
\ifx\paragraph\undefined\else
\let\oldparagraph\paragraph
\renewcommand{\paragraph}[1]{\oldparagraph{#1}\mbox{}}
\fi
\ifx\subparagraph\undefined\else
\let\oldsubparagraph\subparagraph
\renewcommand{\subparagraph}[1]{\oldsubparagraph{#1}\mbox{}}
\fi

%%% Use protect on footnotes to avoid problems with footnotes in titles
\let\rmarkdownfootnote\footnote%
\def\footnote{\protect\rmarkdownfootnote}

%%% Change title format to be more compact
\usepackage{titling}

% Create subtitle command for use in maketitle
\newcommand{\subtitle}[1]{
  \posttitle{
    \begin{center}\large#1\end{center}
    }
}

\setlength{\droptitle}{-2em}

  \title{Income, Energy Consumption and CO2-Emissions with R Notebook}
    \pretitle{\vspace{\droptitle}\centering\huge}
  \posttitle{\par}
    \author{SAB}
    \preauthor{\centering\large\emph}
  \postauthor{\par}
    \date{}
    \predate{}\postdate{}
  

\begin{document}
\maketitle

\begin{center}\rule{0.5\linewidth}{\linethickness}\end{center}

\subsection{Introduction}\label{introduction}

This is an \href{http://rmarkdown.rstudio.com}{R Markdown} Notebook. In
this document, the analysis from the last
\href{https://docs.google.com/spreadsheets/d/1NZpys5yniDyOuUPxV7JKWkcTP-FUZL1esZSvUitZPd0/edit?usp=sharing}{Google
Sheets Document} will be extended by:

\begin{enumerate}
\def\labelenumi{\arabic{enumi}.}
\item
  Visualizing in R the relationships between Income, Energy consumption
  and CO2-emission,
\item
  Incrementing the dataset with further Energy- and CO2-emissions
  related variables, and thereafter expand the previous correlation
  matrix with additional Income-, Energy- and CO2-emissions-related
  variables,
\item
  Performing a Cluster Analysis based on a Multivariate Factor Analysis
  (MFA), a method that aims at synthesizing the information from all
  available data on Income, Energy and CO2-emissions.
\item
  Visualizing the results from the Cluster Analysis on ArcGIS.
\end{enumerate}

The considered dataset was used in an earlier project on
\emph{Sustainable Energy Consumption} - hence not up to date - and was
sourced from the World Bank.

\begin{center}\rule{0.5\linewidth}{\linethickness}\end{center}

\subsection{Premilimary Illustrations}\label{premilimary-illustrations}

To illustrate the revolving idea behind how some variables relate to
each others, i.e.~the notion of relationships or correlation
\emph{(sammenhæng)}, let's start these 2 common examples found on the
net.

Consider first the amount of limonade sold in function of the day's
temperature.

\begin{Shaded}
\begin{Highlighting}[]
\NormalTok{sales <-}\StringTok{ }\KeywordTok{data.frame}\NormalTok{(}\KeywordTok{c}\NormalTok{(}\FloatTok{14.2}\NormalTok{, }\FloatTok{16.4}\NormalTok{, }\FloatTok{11.9}\NormalTok{, }\FloatTok{15.2}\NormalTok{, }\FloatTok{18.5}\NormalTok{, }\FloatTok{22.1}\NormalTok{, }\FloatTok{19.4}\NormalTok{, }\FloatTok{25.1}\NormalTok{, }\FloatTok{23.4}\NormalTok{, }\FloatTok{18.1}\NormalTok{, }\FloatTok{22.6}\NormalTok{, }\FloatTok{17.2}\NormalTok{),}
                    \KeywordTok{c}\NormalTok{(}\DecValTok{215}\NormalTok{, }\DecValTok{325}\NormalTok{, }\DecValTok{185}\NormalTok{, }\DecValTok{332}\NormalTok{, }\DecValTok{406}\NormalTok{, }\DecValTok{522}\NormalTok{, }\DecValTok{412}\NormalTok{, }\DecValTok{614}\NormalTok{, }\DecValTok{544}\NormalTok{, }\DecValTok{421}\NormalTok{, }\DecValTok{445}\NormalTok{, }\DecValTok{408}\NormalTok{))}
\KeywordTok{colnames}\NormalTok{(sales) <-}\StringTok{ }\KeywordTok{c}\NormalTok{(}\StringTok{"Temperature (C)"}\NormalTok{,}\StringTok{"Limonade Sales (Quantity)"}\NormalTok{)}
\NormalTok{sales}
\end{Highlighting}
\end{Shaded}

\begin{verbatim}
##    Temperature (C) Limonade Sales (Quantity)
## 1             14.2                       215
## 2             16.4                       325
## 3             11.9                       185
## 4             15.2                       332
## 5             18.5                       406
## 6             22.1                       522
## 7             19.4                       412
## 8             25.1                       614
## 9             23.4                       544
## 10            18.1                       421
## 11            22.6                       445
## 12            17.2                       408
\end{verbatim}

The values can be represented graphically as below.

\begin{Shaded}
\begin{Highlighting}[]
\KeywordTok{plot}\NormalTok{(sales)}
\end{Highlighting}
\end{Shaded}

\includegraphics{EnergyNotebook_files/figure-latex/unnamed-chunk-2-1.pdf}

As can be seen, there seems to be a relationship between the amount of
limonade sold and the temperature of the day, and a positive one.

In contrast, consider the amount of gazoline left in the tank of a car
as the car travels a certain distance.

\begin{Shaded}
\begin{Highlighting}[]
\NormalTok{tankLevel <-}\StringTok{ }\KeywordTok{data.frame}\NormalTok{(}\KeywordTok{c}\NormalTok{(}\DecValTok{0}\NormalTok{, }\DecValTok{50}\NormalTok{, }\DecValTok{150}\NormalTok{, }\DecValTok{275}\NormalTok{, }\DecValTok{350}\NormalTok{, }\DecValTok{425}\NormalTok{, }\DecValTok{540}\NormalTok{, }\DecValTok{680}\NormalTok{, }\DecValTok{700}\NormalTok{, }\DecValTok{750}\NormalTok{),}
                    \KeywordTok{c}\NormalTok{(}\DecValTok{15}\NormalTok{, }\DecValTok{14}\NormalTok{, }\DecValTok{12}\NormalTok{, }\FloatTok{9.5}\NormalTok{, }\DecValTok{8}\NormalTok{, }\FloatTok{6.5}\NormalTok{, }\FloatTok{4.2}\NormalTok{, }\FloatTok{1.4}\NormalTok{, }\DecValTok{1}\NormalTok{, }\DecValTok{0}\NormalTok{))}
\KeywordTok{colnames}\NormalTok{(tankLevel) <-}\StringTok{ }\KeywordTok{c}\NormalTok{(}\StringTok{"Distance Travelled (km)"}\NormalTok{,}\StringTok{"Gazoline Level (Quantity)"}\NormalTok{)}
\NormalTok{tankLevel}
\end{Highlighting}
\end{Shaded}

\begin{verbatim}
##    Distance Travelled (km) Gazoline Level (Quantity)
## 1                        0                      15.0
## 2                       50                      14.0
## 3                      150                      12.0
## 4                      275                       9.5
## 5                      350                       8.0
## 6                      425                       6.5
## 7                      540                       4.2
## 8                      680                       1.4
## 9                      700                       1.0
## 10                     750                       0.0
\end{verbatim}

Graphically,

\begin{Shaded}
\begin{Highlighting}[]
\KeywordTok{plot}\NormalTok{(tankLevel)}
\end{Highlighting}
\end{Shaded}

\includegraphics{EnergyNotebook_files/figure-latex/unnamed-chunk-4-1.pdf}

Again, there appears to be a relationship between the amount of distance
travelled and the level of gazoline in the tank of the car. Although
this time, even more clearly, the graph demonstrates a negative
relationship.

\begin{center}\rule{0.5\linewidth}{\linethickness}\end{center}

\subsection{Sustainable Energy
Consumption}\label{sustainable-energy-consumption}

Now that the idea of relationship, or \emph{correlation}, has been
introduced, let's shift back to the analysis considered at the beginning
of the document. Among the questions that the analysis will try to
answer, here are a few ones:

\begin{itemize}
\item
  Do average income, energy consumption and CO2-emissions of the
  countries of the world relate to each other?
\item
  Do these different patterns of relationship also hold for other
  income-, energy- and CO2-emissions-related variables?
\item
  Given a region or income level, how (dis)similar are countries in
  terms of Energy Consumption and CO2-emissions?
\end{itemize}

The analysis will be carried out in R and requires the use of the
following packages (a collection of many functions):

\begin{Shaded}
\begin{Highlighting}[]
\KeywordTok{library}\NormalTok{(tidyverse)}
\end{Highlighting}
\end{Shaded}

\begin{verbatim}
FALSE Warning: package 'dplyr' was built under R version 3.5.3
\end{verbatim}

\begin{Shaded}
\begin{Highlighting}[]
\KeywordTok{library}\NormalTok{(PerformanceAnalytics)}
\KeywordTok{library}\NormalTok{(gdata)}
\KeywordTok{library}\NormalTok{(FactoMineR)}
\KeywordTok{library}\NormalTok{(missMDA)}
\KeywordTok{library}\NormalTok{(sp)}
\KeywordTok{library}\NormalTok{(arcgisbinding)}
\end{Highlighting}
\end{Shaded}

To start, let's read the available data on Income, Energy Consumption
and CO2-emissions.

\begin{Shaded}
\begin{Highlighting}[]
\NormalTok{Energy_comp =}\StringTok{ }\KeywordTok{read.csv}\NormalTok{(}\StringTok{"3_TidyData/EnergyCompleted_Final.csv"}\NormalTok{,}
                       \DataTypeTok{header=}\OtherTok{TRUE}\NormalTok{,}\DataTypeTok{sep=}\StringTok{","}\NormalTok{, }\DataTypeTok{dec=}\StringTok{"."}\NormalTok{, }
                       \DataTypeTok{row.names =} \DecValTok{1}\NormalTok{)}
\KeywordTok{dim}\NormalTok{(Energy_comp)}
\end{Highlighting}
\end{Shaded}

\begin{verbatim}
## [1] 147  18
\end{verbatim}

As can be seen, the data consists of 147 countries of the world, the
different regions/categories they belong to and the 3 afore-mentioned
variable groups with 18 columns in total.

Let's have a preview of the dataset:

\begin{Shaded}
\begin{Highlighting}[]
\KeywordTok{head}\NormalTok{(Energy_comp, }\DecValTok{10}\NormalTok{)}
\end{Highlighting}
\end{Shaded}

\begin{verbatim}
##             Country.Name                     Region        Income.Group
## AFG          Afghanistan                 South Asia          Low income
## AGO               Angola         Sub-Saharan Africa Lower middle income
## ALB              Albania      Europe & Central Asia Upper middle income
## ARE United Arab Emirates Middle East & North Africa         High income
## ARG            Argentina  Latin America & Caribbean Upper middle income
## ARM              Armenia      Europe & Central Asia Lower middle income
## AUS            Australia        East Asia & Pacific         High income
## AUT              Austria      Europe & Central Asia         High income
## AZE           Azerbaijan      Europe & Central Asia Upper middle income
## BDI              Burundi         Sub-Saharan Africa          Low income
##     Population GDP..Mio.USD. GDP.per.capita   HDI
## AFG   34656032       65142.8         1802.7 0.479
## AGO   28813463      186327.0         5984.6 0.533
## ALB    2876101       34126.4        11359.2 0.764
## ARE    9269612      672419.6        67133.1 0.840
## ARG   43847430      876012.1        18489.4 0.827
## ARM    2924816       25884.4         8190.2 0.743
## AUS   24127159     1128908.0        44260.6 0.939
## AUT    8747358      443005.0        44438.7 0.893
## AZE    9762274      168713.8        16001.3 0.759
## BDI   10524117        8201.1          721.2 0.404
##     Electricity.Consumption.per.capita..kWh
## AFG                                   10.00
## AGO                                  312.48
## ALB                                 2309.37
## ARE                                11263.53
## ARG                                 3052.38
## ARM                                 1965.78
## AUS                                10059.21
## AUT                                 8360.52
## AZE                                 2202.39
## BDI                                   10.00
##     Fossil.fuel.energy.consumption..pct..of.total.
## AFG                                       56.31438
## AGO                                       48.28000
## ALB                                       61.42000
## ARE                                       99.81000
## ARG                                       88.54000
## ARM                                       74.56000
## AUS                                       93.39000
## AUT                                       64.90000
## AZE                                       98.37000
## BDI                                       55.73222
##     Electricity.production.from.oil.sources..pct..of.total.
## AFG                                                24.65847
## AGO                                                46.82000
## ALB                                                 0.00000
## ARE                                                 1.34000
## ARG                                                13.83000
## ARM                                                 0.00000
## AUS                                                 2.02000
## AUT                                                 0.99000
## AZE                                                 0.16000
## BDI                                                27.22136
##     Electricity.production.from.oil.gas.coal.sources..pct.of.total.
## AFG                                                        57.28404
## AGO                                                        46.82000
## ALB                                                         0.00000
## ARE                                                        99.73000
## ARG                                                        64.39000
## ARM                                                        42.44000
## AUS                                                        85.09000
## AUT                                                        17.73000
## AZE                                                        94.02000
## BDI                                                        59.32039
##     Electricity.production.from.hydroelectric.sources..pct.of.total.
## AFG                                                         37.86226
## AGO                                                         53.18000
## ALB                                                        100.00000
## ARE                                                          0.00000
## ARG                                                         29.04000
## ARM                                                         25.70000
## AUS                                                          7.41000
## AUT                                                         66.57000
## AZE                                                          5.26000
## BDI                                                         37.62264
##     Electricity.production.from.renewables.sources...excl..hydroelectric..pct.of.total.
## AFG                                                                            3.689814
## AGO                                                                            0.000000
## ALB                                                                            0.000000
## ARE                                                                            0.270000
## ARG                                                                            2.500000
## ARM                                                                            0.050000
## AUS                                                                            7.500000
## AUT                                                                           14.560000
## AZE                                                                            0.370000
## BDI                                                                            2.937671
##     Alternative.and.nuclear.energy..pct..of.total.energy.use.
## AFG                                                  5.081635
## AGO                                                  2.950000
## ALB                                                 17.920000
## ARE                                                  0.100000
## ARG                                                  5.880000
## ARM                                                 27.510000
## AUS                                                  2.550000
## AUT                                                 12.870000
## AZE                                                  0.780000
## BDI                                                  3.303454
##     Energy.use.per.capita..kg.oil.eq. CO2.emissions..kt.
## AFG                             50.00            9809.23
## AGO                            545.04           34763.16
## ALB                            808.46            5716.85
## ARE                           7769.23          211369.55
## ARG                           2015.19          204024.55
## ARM                           1018.07            5529.84
## AUS                           5328.22          361261.84
## AUT                           3765.43           58712.34
## AZE                           1502.08           37487.74
## BDI                             50.00             440.04
##     CO2.emissions..tons.per.capita.
## AFG                            0.30
## AGO                            1.29
## ALB                            1.98
## ARE                           23.30
## ARG                            4.75
## ARM                            1.90
## AUS                           15.37
## AUT                            6.87
## AZE                            3.93
## BDI                            0.04
##     CO2.emissions.from.transport..pct.of.total.fuel.combustion.
## AFG                                                    39.70712
## AGO                                                    43.99000
## ALB                                                    59.95000
## ARE                                                    21.04000
## ARG                                                    24.17000
## ARM                                                    27.78000
## AUS                                                    24.74000
## AUT                                                    36.62000
## AZE                                                    24.49000
## BDI                                                    40.07004
\end{verbatim}

A summary of the dataset follows:

\begin{Shaded}
\begin{Highlighting}[]
\KeywordTok{print}\NormalTok{(}\KeywordTok{summary}\NormalTok{(Energy_comp))}
\end{Highlighting}
\end{Shaded}

\begin{verbatim}
##       Country.Name                        Region  
##  Namibia    :  2   East Asia & Pacific       :17  
##  Afghanistan:  1   Europe & Central Asia     :47  
##  Albania    :  1   Latin America & Caribbean :23  
##  Algeria    :  1   Middle East & North Africa:17  
##  Angola     :  1   North America             : 2  
##  Argentina  :  1   South Asia                : 7  
##  (Other)    :140   Sub-Saharan Africa        :34  
##               Income.Group   Population        GDP..Mio.USD.     
##  High income        :46    Min.   :3.195e+04   Min.   :     100  
##  Low income         :23    1st Qu.:4.733e+06   1st Qu.:   38279  
##  Lower middle income:36    Median :1.065e+07   Median :  153199  
##  Upper middle income:42    Mean   :4.801e+07   Mean   :  796355  
##                            3rd Qu.:3.547e+07   3rd Qu.:  510809  
##                            Max.   :1.379e+09   Max.   :21450968  
##                                                                  
##  GDP.per.capita        HDI         Electricity.Consumption.per.capita..kWh
##  Min.   :   100   Min.   :0.3520   Min.   :   10.0                        
##  1st Qu.:  5219   1st Qu.:0.6115   1st Qu.:  579.2                        
##  Median : 13921   Median :0.7360   Median : 2202.4                        
##  Mean   : 19788   Mean   :0.7100   Mean   : 3911.1                        
##  3rd Qu.: 26569   3rd Qu.:0.8285   3rd Qu.: 5032.5                        
##  Max.   :118207   Max.   :0.9490   Max.   :53832.5                        
##                                                                           
##  Fossil.fuel.energy.consumption..pct..of.total.
##  Min.   :  5.36                                
##  1st Qu.: 54.28                                
##  Median : 67.99                                
##  Mean   : 65.89                                
##  3rd Qu.: 85.48                                
##  Max.   :100.00                                
##                                                
##  Electricity.production.from.oil.sources..pct..of.total.
##  Min.   : 0.00                                          
##  1st Qu.: 0.35                                          
##  Median : 2.84                                          
##  Mean   :15.04                                          
##  3rd Qu.:22.68                                          
##  Max.   :99.46                                          
##                                                         
##  Electricity.production.from.oil.gas.coal.sources..pct.of.total.
##  Min.   :  0.00                                                 
##  1st Qu.: 38.20                                                 
##  Median : 58.02                                                 
##  Mean   : 57.88                                                 
##  3rd Qu.: 86.47                                                 
##  Max.   :100.00                                                 
##                                                                 
##  Electricity.production.from.hydroelectric.sources..pct.of.total.
##  Min.   :  0.000                                                 
##  1st Qu.:  4.495                                                 
##  Median : 27.610                                                 
##  Mean   : 30.089                                                 
##  3rd Qu.: 41.600                                                 
##  Max.   :100.000                                                 
##                                                                  
##  Electricity.production.from.renewables.sources...excl..hydroelectric..pct.of.total.
##  Min.   : 0.000                                                                     
##  1st Qu.: 0.125                                                                     
##  Median : 3.056                                                                     
##  Mean   : 6.499                                                                     
##  3rd Qu.: 7.235                                                                     
##  Max.   :55.830                                                                     
##                                                                                     
##  Alternative.and.nuclear.energy..pct..of.total.energy.use.
##  Min.   : 0.00                                            
##  1st Qu.: 1.33                                            
##  Median : 5.11                                            
##  Mean   :10.22                                            
##  3rd Qu.:12.29                                            
##  Max.   :91.98                                            
##                                                           
##  Energy.use.per.capita..kg.oil.eq. CO2.emissions..kt.
##  Min.   :   50.0                   Min.   :      50  
##  1st Qu.:  542.3                   1st Qu.:    6302  
##  Median : 1337.4                   Median :   26450  
##  Mean   : 2213.2                   Mean   :  224420  
##  3rd Qu.: 2771.2                   3rd Qu.:   95844  
##  Max.   :18562.7                   Max.   :10291927  
##                                                      
##  CO2.emissions..tons.per.capita.
##  Min.   : 0.010                 
##  1st Qu.: 0.840                 
##  Median : 3.050                 
##  Mean   : 4.786                 
##  3rd Qu.: 6.220                 
##  Max.   :45.420                 
##                                 
##  CO2.emissions.from.transport..pct.of.total.fuel.combustion.
##  Min.   : 3.54                                              
##  1st Qu.:23.60                                              
##  Median :34.78                                              
##  Mean   :34.05                                              
##  3rd Qu.:40.51                                              
##  Max.   :96.78                                              
## 
\end{verbatim}

\textbf{PS}: the existence of missing values for some variables was
dealt with imputation methods from the package \emph{missMDA}. Suffice
to write here that the original dataset was trimmed of some countries
that lack many values, while at the same time, 37 countries were kept
for the analysis by imputing some of their missing values using the
correlation structure of the dataset. Refer to this
\href{https://arcg.is/uW4Ky}{App} to compare the analysis with complete
observations (110) and the one based on the augmented dataset (the
current one) with imputed observations (147).

\begin{center}\rule{0.5\linewidth}{\linethickness}\end{center}

\subsection{Simple correlations from the lecture's
dataset}\label{simple-correlations-from-the-lectures-dataset}

With this dataset, one can visualize the few correlations introduced in
the
\href{https://docs.google.com/spreadsheets/d/1NZpys5yniDyOuUPxV7JKWkcTP-FUZL1esZSvUitZPd0/edit?usp=sharing}{Google
Sheets Document}. This is done with the package
\emph{PerformanceAnalytics} as follows.

\begin{Shaded}
\begin{Highlighting}[]
\KeywordTok{chart.Correlation}\NormalTok{(Energy_comp[,}\KeywordTok{c}\NormalTok{(}\DecValTok{6}\NormalTok{,}\DecValTok{15}\NormalTok{,}\DecValTok{17}\NormalTok{)], }\DataTypeTok{histogram=}\OtherTok{TRUE}\NormalTok{, }\DataTypeTok{pch=}\DecValTok{19}\NormalTok{)}
\end{Highlighting}
\end{Shaded}

\includegraphics{EnergyNotebook_files/figure-latex/unnamed-chunk-9-1.pdf}

Focusing first on the scatterplots, the graphical matrix could be read
as one read for example a multiplication table, that is reading the rows
first and then the columns: There, were the row belonging to
\emph{Energy Use per Capita} meets the column of \emph{GDP per Capita},
the scatter-plot suggests a rather pronounced correlation between these
2 variables, although the correlation between \emph{Energy Use per
Capita} and \emph{CO2 Emissions per Capita} appears to be slightly more
significant. Not surprisingly given the well-known fact that fossile
fuels still account for about 80 pct. of total energy consumed on the
global level, far ahead of the share of renewable energy sources (around
10 pct.) as discussed during the lectures.

The 3 scatterplots can be drawn once more, this time visualizing the
income groups each country (point) pertains to:

\begin{Shaded}
\begin{Highlighting}[]
\NormalTok{lower.panel<-}\ControlFlowTok{function}\NormalTok{(x, y)\{}
  \KeywordTok{points}\NormalTok{(x,y, }\DataTypeTok{pch=}\DecValTok{19}\NormalTok{, }\DataTypeTok{col=}\KeywordTok{c}\NormalTok{(}\StringTok{"red"}\NormalTok{, }\StringTok{"green"}\NormalTok{, }\StringTok{"blue"}\NormalTok{, }\StringTok{"grey"}\NormalTok{)[Energy_comp}\OperatorTok{$}\NormalTok{Income.Group])}
\NormalTok{  r <-}\StringTok{ }\KeywordTok{round}\NormalTok{(}\KeywordTok{cor}\NormalTok{(x, y), }\DataTypeTok{digits=}\DecValTok{2}\NormalTok{)}
\NormalTok{  txt <-}\StringTok{ }\KeywordTok{paste0}\NormalTok{(}\StringTok{"R = "}\NormalTok{, r)}
\NormalTok{  usr <-}\StringTok{ }\KeywordTok{par}\NormalTok{(}\StringTok{"usr"}\NormalTok{); }\KeywordTok{on.exit}\NormalTok{(}\KeywordTok{par}\NormalTok{(usr))}
  \KeywordTok{par}\NormalTok{(}\DataTypeTok{usr =} \KeywordTok{c}\NormalTok{(}\DecValTok{0}\NormalTok{, }\DecValTok{1}\NormalTok{, }\DecValTok{0}\NormalTok{, }\DecValTok{1}\NormalTok{))}
  \KeywordTok{text}\NormalTok{(}\FloatTok{0.5}\NormalTok{, }\FloatTok{0.9}\NormalTok{, txt)}
\NormalTok{\}}
\KeywordTok{pairs}\NormalTok{(Energy_comp[,}\KeywordTok{c}\NormalTok{(}\DecValTok{6}\NormalTok{,}\DecValTok{15}\NormalTok{,}\DecValTok{17}\NormalTok{)], }\DataTypeTok{upper.panel =} \OtherTok{NULL}\NormalTok{, }
      \DataTypeTok{lower.panel =}\NormalTok{ lower.panel)}
\end{Highlighting}
\end{Shaded}

\includegraphics{EnergyNotebook_files/figure-latex/unnamed-chunk-10-1.pdf}
\textbf{PS}: The colour conventions are:

\begin{itemize}
\tightlist
\item
  \textbf{Red} for \emph{high income} countries
\item
  \textbf{Green} for \emph{low income} countries
\item
  \textbf{Blue} for \emph{lower middle income} countries
\item
  \textbf{Grey} for \emph{upper middle income} countries
\end{itemize}

\begin{center}\rule{0.5\linewidth}{\linethickness}\end{center}

\subsection{Extended Correlation
Structure}\label{extended-correlation-structure}

In the same fashion, one can expand the correlation matrix to include
the remainder of the continuous variables at hand:

\begin{Shaded}
\begin{Highlighting}[]
\KeywordTok{chart.Correlation}\NormalTok{(Energy_comp[,}\KeywordTok{c}\NormalTok{(}\DecValTok{5}\OperatorTok{:}\DecValTok{18}\NormalTok{)], }\DataTypeTok{histogram=}\OtherTok{TRUE}\NormalTok{, }\DataTypeTok{pch=}\DecValTok{19}\NormalTok{)}
\end{Highlighting}
\end{Shaded}

\includegraphics{EnergyNotebook_files/figure-latex/unnamed-chunk-11-1.pdf}

Given the difficulty to distinguish between the details in such a
graphical illustration, one can list the strongest correlations among
all variables. This have been carried out in the next code chunk. In
passing, note that the strongest relationships in the matrix are
suggested by the number of red stars (3 means likelier relationships) or
the numerical value of the correlation coefficient (stronger
correlations where the coefficient is closer to 1 in absolute terms).
Any biasing effects from outliers have been ignored.

\begin{Shaded}
\begin{Highlighting}[]
\NormalTok{corm <-}\StringTok{ }\KeywordTok{cor}\NormalTok{(Energy_comp[,}\KeywordTok{c}\NormalTok{(}\DecValTok{5}\OperatorTok{:}\DecValTok{18}\NormalTok{)])}
\NormalTok{corm[}\KeywordTok{lower.tri}\NormalTok{(corm)] <-}\StringTok{ }\DecValTok{0}
\NormalTok{corm[}\KeywordTok{lower.tri}\NormalTok{(corm,}\DataTypeTok{diag=}\OtherTok{TRUE}\NormalTok{)] <-}\StringTok{ }\DecValTok{0}
\NormalTok{cor <-}\StringTok{ }\KeywordTok{as.data.frame}\NormalTok{(}\KeywordTok{as.table}\NormalTok{(corm))}
\NormalTok{high<-}\KeywordTok{subset}\NormalTok{(cor, }\KeywordTok{abs}\NormalTok{(Freq) }\OperatorTok{>}\StringTok{ }\FloatTok{0.6}\NormalTok{)}
\KeywordTok{as.matrix}\NormalTok{(high[}\KeywordTok{order}\NormalTok{(}\OperatorTok{-}\NormalTok{high[,}\DecValTok{3}\NormalTok{]),])}
\end{Highlighting}
\end{Shaded}

\begin{verbatim}
##     Var1                                                             
## 155 "GDP..Mio.USD."                                                  
## 144 "Electricity.Consumption.per.capita..kWh"                        
## 179 "Energy.use.per.capita..kg.oil.eq."                              
## 170 "GDP.per.capita"                                                 
## 142 "GDP.per.capita"                                                 
## 30  "GDP.per.capita"                                                 
## 44  "GDP.per.capita"                                                 
## 89  "Fossil.fuel.energy.consumption..pct..of.total."                 
## 187 "Fossil.fuel.energy.consumption..pct..of.total."                 
## 133 "Electricity.production.from.oil.gas.coal.sources..pct.of.total."
## 105 "Electricity.production.from.oil.gas.coal.sources..pct.of.total."
##     Var2                                                              
## 155 "CO2.emissions..kt."                                              
## 144 "Energy.use.per.capita..kg.oil.eq."                               
## 179 "CO2.emissions..tons.per.capita."                                 
## 170 "CO2.emissions..tons.per.capita."                                 
## 142 "Energy.use.per.capita..kg.oil.eq."                               
## 30  "HDI"                                                             
## 44  "Electricity.Consumption.per.capita..kWh"                         
## 89  "Electricity.production.from.oil.gas.coal.sources..pct.of.total." 
## 187 "CO2.emissions.from.transport..pct.of.total.fuel.combustion."     
## 133 "Alternative.and.nuclear.energy..pct..of.total.energy.use."       
## 105 "Electricity.production.from.hydroelectric.sources..pct.of.total."
##     Freq        
## 155 " 0.9527348"
## 144 " 0.8760625"
## 179 " 0.8594627"
## 170 " 0.8343659"
## 142 " 0.8151981"
## 30  " 0.7420396"
## 44  " 0.6840740"
## 89  " 0.6123586"
## 187 "-0.6460173"
## 133 "-0.6515375"
## 105 "-0.8331979"
\end{verbatim}

The correlation coefficients are arranged in descending order, and the
ones of interest are those that are closest to 1 in absolute values.

Hence, there seems to be further relevant correlations among the
variables from the extended dataset. To name a few \ldots{}

\begin{itemize}
\item
  On the positive side:

  \begin{itemize}
  \item
    Total income and total CO2-emissions,
  \item
    Electricity Consumption and Energy Use per capita,
  \end{itemize}
\item
  On the negative side:

  \begin{itemize}
  \item
    Electricity production from fossile fuels and from hydroelectric
    sources
  \item
    Electricity production from fossile fuels and from alternative and
    nuclear energy
  \end{itemize}
\end{itemize}

This seems in line with general intuition and corroborates earlier
observations.

\begin{center}\rule{0.5\linewidth}{\linethickness}\end{center}

\subsection{Hierarchical Clustering}\label{hierarchical-clustering}

Moving now the focus towards the study of the many countries of the
world and how (dis)similar they are with respect to each other in term
of income, energy use and CO2-emissions, the many variables at hand can
be synthethised to help us better answer a few additional questions:

\begin{itemize}
\item
  Are there group of countries that shows the same pattern of income
  level, energy consumption and CO2-emissions?
\item
  If so, which cluster of countries shows similarities or, by contrast,
  dissimilarities in terms of income level, energy consumption and
  CO2-emissions?
\end{itemize}

These questions and many others can be answered by means of a Cluster
Analysis, and the chosen method is hierarchical clustering, the one that
by default is provided by the library \emph{FactoMineR}.

Prior to this, a Multivariate Factor Analysis was undertaken so as to
reduce the multidimensionality of the dataset, but at the same time, to
take into account the grouped nature of the dataset's variables. This
can be helpful when it comes to group the 147 countries into different
clusters and characterise each resulting clusters by means of the
synthetic variables that emerge from such an analysis.

The detailed outline of this analysis is skipped, but the relevant code
chunks are provided as follows:

\begin{Shaded}
\begin{Highlighting}[]
\NormalTok{res <-}\StringTok{ }\KeywordTok{MFA}\NormalTok{(Energy_comp[,}\KeywordTok{c}\NormalTok{(}\DecValTok{2}\OperatorTok{:}\DecValTok{18}\NormalTok{)], }\DataTypeTok{group=}\KeywordTok{c}\NormalTok{(}\DecValTok{1}\NormalTok{,}\DecValTok{1}\NormalTok{,}\DecValTok{2}\NormalTok{,}\DecValTok{2}\NormalTok{,}\DecValTok{8}\NormalTok{,}\DecValTok{3}\NormalTok{), }\DataTypeTok{type=}\KeywordTok{c}\NormalTok{(}\KeywordTok{rep}\NormalTok{(}\StringTok{"n"}\NormalTok{,}\DecValTok{2}\NormalTok{),}\KeywordTok{rep}\NormalTok{(}\StringTok{"s"}\NormalTok{,}\DecValTok{4}\NormalTok{)),}
           \DataTypeTok{ncp=}\DecValTok{4}\NormalTok{, }\DataTypeTok{graph=}\NormalTok{F, }\DataTypeTok{name.group=}\KeywordTok{c}\NormalTok{(}\StringTok{"Region"}\NormalTok{,}\StringTok{"IncomeGroup"}\NormalTok{,}\StringTok{"Desc"}\NormalTok{,}\StringTok{"Income"}\NormalTok{,}\StringTok{"Energy"}\NormalTok{,}\StringTok{"Emmissions"}\NormalTok{),}
           \DataTypeTok{num.group.sup=}\KeywordTok{c}\NormalTok{(}\DecValTok{2}\OperatorTok{:}\DecValTok{3}\NormalTok{))}
\end{Highlighting}
\end{Shaded}

\begin{Shaded}
\begin{Highlighting}[]
\NormalTok{res.hcpc <-}\StringTok{ }\KeywordTok{HCPC}\NormalTok{(res, }\DataTypeTok{kk =} \OtherTok{Inf}\NormalTok{, }\DataTypeTok{graph=}\NormalTok{F)}
\end{Highlighting}
\end{Shaded}

Once the different clusters have been obtained (their numbers specified
with some degree of subjectivity, see later), these can be transferred
and mapped in
\href{https://sanktpetriskole.maps.arcgis.com/apps/View/index.html?appid=39999e464aa94b528d7eb0d2c5543561}{ArcGIS}.

Here is also a
\href{https://drive.google.com/open?id=1vyLpeukQfa64Bq5P0KbWMXfSb2s_foBj}{screenshot}
of the resulting clustering.

When it comes to interpretate the resulting clustering, the different
outputs from the above 2 code chunks are handy. Keeping it short, among
the main observations that can be derived from the outputs:

\begin{itemize}
\item
  \textbf{Cluster 5}: Countries with respectively high income, energy
  use and CO2-emissions per capita.
\item
  \textbf{Cluster 1}: Countries with respectively low income, energy use
  and CO2-emissions per capita.
\item
  \textbf{Cluster 4}: Countries with high total income, energy use and
  CO2-emissions.
\item
  \textbf{Cluster 3}: Countries with relatively:

  \begin{itemize}
  \item
    high proportion of alternative and nuclear energy content in their
    total energy use; high proportion of electricity production from
    hydroelectric sources; high proportion of CO2-emissions from
    transport as a percentage of total fuel combustion; high electricity
    consumption per capita.
  \item
    Low proportion of electricity production from fossil fuel sources;
    low proportion of fossil fuel energy consumption as a percentage of
    total energy use.
  \end{itemize}
\item
  \textbf{Cluster 2}: In contrast with \textbf{Cluster 3}, these are
  countries with relatively:

  \begin{itemize}
  \item
    high proportion of fossil fuel energy consumption as a percentage of
    total energy use; high proportion of electricity production from
    fossil fuel sources; high HDI (Human Development Index).
  \item
    low proportion of CO2-emissions from transport as a percentage of
    total fuel combustion; low proportion of electricity production from
    hydroelectric sources; low proportion of alternative and nuclear
    energy content in their total energy use;
  \end{itemize}
\end{itemize}

In the same token, focusing on the 2 categorical variables of the
analysis, namely \emph{Regions} and \emph{IncomeGroup}, the following
observations can be made from the code output:

\begin{itemize}
\item
  \textbf{Cluster 1} tend to be overrepresented by \emph{Sub-Saharan}
  and \emph{Low Income} countries.
\item
  \textbf{Cluster 2} contains a higher proportion of \emph{Upper Middle
  Income} countries and \emph{European/Central Asian} countries.
\item
  \textbf{Cluster 3} are all \emph{High Income} countries, most of which
  are \emph{European/Central Asian}.
\item
  \textbf{Cluster 5} are all \emph{High Income} countries, most of them
  belonging to the \emph{Middle East and North African} regions.
\end{itemize}

\begin{center}\rule{0.5\linewidth}{\linethickness}\end{center}

\subsection{Final thoughts}\label{final-thoughts}

Cluster analysis, along with Multivariate Factor Analysis, are methods
used to synthethise available information in a given dataset that
usually contains more than 10 variables. From this preliminary analysis:

\begin{itemize}
\item
  Most of the results - notwithstanding the outdated nature of some
  variables (a few dating back from 2011) - are in line with what one
  would expect from the relationships between Income, Energy Consumption
  and CO2-emissions.
\item
  A few countries have been surprisingly clustered with others
  e.g.~across different income groups, something that seems to
  contradict common understanding. Given the use of MFA as the
  preliminary tool for dimensionality reduction, these artifacts, that
  by themselves deserve checking the original dataset and/or further
  investigations, could a priori be justified by the rebalancing nature
  of the MFA-methodology: no group among the afore-mentioned 3 is
  allowed to outweight the other 2 when it comes to define the primary
  components of variability in the dataset. A cluster analysis based on
  a standard PCA would have resulted in different clusterings, arguably
  more in line with common intuition.
\item
  The defined numbers of clustering is to some extent a matter of
  subjectivity (albeit a few criteria exists to optimally define the
  right ratio between inter and total inertia), and a different level of
  clustering would probably have made this analysis either too
  simplistic (loosing some interesting insigths, which this analysis has
  already done by only accounting for the first 2 dimensions of
  variability representing slightly above 50 pct. of all information in
  the dataset) or too detailed to easily discern common patterns between
  the dataset's observations.
\item
  These limitations and other warrants further study of such an
  interesting topic.
\end{itemize}


\end{document}
